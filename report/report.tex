\documentclass[a4paper]{article}
\usepackage[utf8]{inputenc}
\usepackage[francais]{babel}
\usepackage[T1]{fontenc}
\usepackage{lmodern}
\usepackage{fullpage}
\usepackage{amsmath}
\usepackage{graphicx}
\usepackage{framed}
\usepackage{listings}
\usepackage{placeins}
\usepackage{subcaption}
\usepackage{array}
\usepackage[justification=centering]{caption}

\author{Théotime Grohens}
\title{Projet : Lustre Model Checker}

\begin{document}

\maketitle

\section{Travail réalisé}

Pour ce projet, j'ai implémenté un model checker Lustre basé sur la $k$-induction, tel que décrit dans la thèse de George Hagen.
Le principe est d'essayer de prouver que la sortie d'un nœud Lustre donnée vaut toujours \texttt{true}.
Pour cela, on traduit le programme Lustre en un ensemble $\Delta(n)$ de formules logiques décrivant l'état des variables du programme au temps $n$ en fonction de leurs valeurs précédentes, et on essaye de prouver à l'aide d'un solveur SMT la propriété $P(n)$ : la variable de sortie vaut \texttt{true}, pour tout $n$.

Le principe de la $k$-induction est de généraliser le principe de récurrence.
Faire une récurrence pour démontrer $P(n)$ revient à montrer $\Delta(0)\models P(0)$ (l'initialisation), et $\Delta(n), P(n) \models P(n+1)$ (l'hérédité).

Cette technique est sûre : si elle parvient à prouver ces deux implications à l'aide du solveur, alors on a démontré que pour tout $n$, $P(n)$ est vraie.
Si le cas de base est faux, elle nous fournit de plus un contre-exemple à la propriété, c'est-à-dire une trace d'exécution pour laquelle la propriété n'est pas vérifiée.
Elle n'est toutefois pas complète : il existe des programmes qui vérifient la propriété, mais pour laquelle on n'arrive pas à prouver l'hérédité.

Pour améliorer la méthode, on essaye donc de faire une récurrence qui s'appuie sur les $k$ précédentes valeurs de $\Delta$ (et donc les $k$ derniers états du programme), au lieu d'utiliser juste le dernier.
On essaye donc de prouver comme cas de base que $\Delta(0), \ldots, \Delta(k) \models P(0) \wedge \ldots \wedge P(k)$, et comme cas inductif que $\Delta(n), \ldots, \Delta(n+k+1), P(n), \ldots, P(n+k) \models P(n+1)$. 

L'idée est que le cas inductif sera plus facile à prouver si l'on fait plus d'hypothèses (sur les états précédents du programme); pour être capable de faire ces hypothèses supplémentaires, il faut toutefois prouver un cas de base plus puissant. 

L'algorithme du model checker est donc simple : il essaie de prouver le programme par $k$-induction, avec $k$ croissant, jusqu'à soit prouver cas de base et cas inductif, soit trouver un cas de base faux, soit atteindre un $k_{max}$ après lequel on abandonne.

\section{Choix techniques}

Mon programme utilise le solveur SMT Z3.
Les étapes de l'algorithme sont les suivantes :
\begin{itemize}
\item \emph{Inliner} les nœuds Lustre appelés par le nœud à vérifier, en prenant garde à renommer les variables pour éviter les conflits
\item Créer des flux auxiliaires afin de supprimer les constructeurs \texttt{pre} imbriqués, de sorte que l'état des flux au temps $n$ ne dépende explicitement que du temps $n-1$
\item Transformer récursivement les parties droites des équations (expressions) en termes du solveur
\item Construire $\Delta(n)$, la conjonction des définitions de variables \texttt{x = e} vues comme des formules, et $P(n)$
\item Essayer de prouver les cas de base et cas inductif de la $k$-induction, pour $k$ allant de 0 à $k_{max}$
\end{itemize}

\pagebreak

\section{Extension}

En plus de l'algorithme de base, j'ai implanté l'extension de compression de chemins décrite dans la thèse.
Celle-ci rend le $k$-model checker complet pour les systèmes ayant au plus $k$ états distincts.

Cependant, cette extension n'est pas suffisante pour prouver que le programme \texttt{counter.lus} est correct, car le checker ne devine pas le fait que la variable \texttt{time} est à valeurs dans $\{0, \ldots, 3\}$ (et donc que l'ensemble des états est bien fini).

\section{Résultats}

Le model checker arrive à prouver des exemples simples ne nécessitant que de la 0-induction (soit une récurrence simple), mais aussi des exemples plus complexes nécessitant d'utiliser des $k$ non nuls.
Par exemple, le programme \texttt{ex003.lus} nécessite une 1-induction.


\end{document}